\documentclass[12pt]{article}
\usepackage{url}
\renewcommand{\rmdefault}{ptm} % What does it do? Somehow resets the default font...
%\usepackage{times}
\usepackage{helvet}

% Pass option resetlabels to start each bibliography with [1]
\usepackage{multibib}
\newcites
{libmesh,theses,preprints,thirteen,twelve,eleven,ten,nine,eight,seven,six,five,four}
{LibMesh Paper,Dissertations \& Theses,Preprints,2013,2012,2011,2010,2009,2008,2007,2006,2005,2004}



\title{Papers by people using LibMesh}
\author{Various}
\begin{document}
\maketitle

% Cite everything
% \nocite*

% Cite everything from 2009
% \cite{Barone_2009b,Blome_2009,Kirk_2009b,Class_2009,
%       Knezevic_2009d,Lu_2009b,AufderMaur_2009b,
%       AufderMaur_2009,Antiga_2009,Knezevic_2009,Lu_2009,
%       Barbarosie_2009,Xu_2009,Polidori_2009,Gaston_2009b,
%       Jones_2009,Mahadevan_2009,Gaston_2009,Luethi2009,
%       Valli_2009,Barone_2009,Biermann_2009,Kirk_2009}

% \bibliographystyle{ieeetr}
% \bibliography{libmesh}


% In the ieeetr format, if you specify a month and an issue number,
% it prints only the month!

% The unsrt format, on the other hand, prints both, like this:
% International Journal of Blah, 13(4):999-1016,November 2009.

% Multiple-bibliography files
\nocitelibmesh*
\bibliographystylelibmesh{plain}
\bibliographylibmesh{libmesh}

\nocitetheses*
\bibliographystyletheses{plain}
\bibliographytheses{theses}

\nocitepreprints*
\bibliographystylepreprints{plain}
\bibliographypreprints{preprints}

% Apparently these names cannot have numbers in them?
\nocitethirteen*
\bibliographystylethirteen{plain}
\bibliographythirteen{thirteen}

\nocitetwelve*                     % for new year, change number here
\bibliographystyletwelve{plain}    % ... and here
\bibliographytwelve{twelve}        % ... and here

\nociteeleven*
\bibliographystyleeleven{plain}
\bibliographyeleven{eleven}

\nociteten*
\bibliographystyleten{plain}
\bibliographyten{ten}

\nocitenine*
\bibliographystylenine{plain}
\bibliographynine{nine}

\nociteeight*
\bibliographystyleeight{plain}
\bibliographyeight{eight}

\nociteseven*
\bibliographystyleseven{plain}
\bibliographyseven{seven}

\nocitesix*
\bibliographystylesix{plain}
\bibliographysix{six}

\nocitefive*
\bibliographystylefive{plain}
\bibliographyfive{five}

\nocitefour*
\bibliographystylefour{plain}
\bibliographyfour{four}




\end{document}
