\documentclass{article}
\usepackage{listings}

\newcommand{\exmp}[1]{\textsf{Ex#1}}
\newcommand{\code}[1]{\texttt{#1}}

\parindent0ex
\oddsidemargin 0in
\evensidemargin 0in
\headheight 2\baselineskip
\textwidth 6.5in
\textheight 9.0in

\begin{document}
\lstset{language=C++}


\title{Libmesh HOWTO}
\author{\texttt{libmesh-users@lists.sourceforge.net} \\
        $$Revision$$}
\maketitle

\section{General usage}
\label{sec:usage-tips}

\subsection{Debug / Profile mode}
\label{sec:debug-profile}

Debugging and profiler modes can be switched on with \code{make METHOD=dbg}
and \code{make METHOD=pro}.  For syntax checking use \code{make METHOD=syn}.

\subsection{Performance logging}
\label{sec:performance-logging}

\begin{itemize}
\item Create a logger: \code{PerfLog perf\_log ("Matrix Assembly")} \exmp{4}.
\item Start logging: \code{perf\_log.start\_event("elem init")}
\item Stop logging: \code{perf\_log.stop\_event("elem init");}
\end{itemize}

\subsection{Petsc-Tools}
\label{sec:petsc-tools}

Use the Petsc tools as command line parameters to the program invocation,
e.g.~\code{./myTestProgram -log\_summary}. Frequently used options are

\begin{itemize}
\item -log\_summary: show setup and performance
\item -log\_info: show setup
\item -ksp\_monitor: show convergence
\end{itemize}

\section{Basic tasks}
\label{sec:basic-tasks}

\subsection{Restart a model}
\label{sec:restart}

Restarting a model is done by loading a simulation, that has been stored with 
\code{equation\_systems.write}. It is stored either in a ASCII-File or in the
HDF format. It is read with  \code{equation\_systems.read}. (\exmp{2}).

\subsection{Translate / deform / rotate a mesh}
\label{sec:translate}

A mesh object can be translated, deformed or rotated with
\begin{lstlisting}
  MeshTools::Modification::translate(mesh, 10., 1.);
  MeshTools::Modification::rotate(mesh, 90., 10. 0.);
\end{lstlisting}

\subsection{Equation system parameters}
\label{sec:EQSParameter}

Equation system parameters are set with the following methods\\[1ex]
\lstinline!es.parameters.set<Real> ("myParam") = 42.;!. \\
\lstinline!es.parameters.set<unsigned int> ("linear solver maximum iterations") = 250;! \\[1ex]
Their values can later be obtained with \\[1ex]
\lstinline!Real answer = es.parameters.get<Real> ("myParam");!. \\

\subsection{Write to postprocessing file}
\label{sec:postprocessing}

Libmesh supports many post-processing file types. Writing the mesh is as easy as\\[1ex]
\lstinline!GMVIO(mesh).write("out.gmv");!\\[1ex]
Writing a the mesh together with the current solution is equally simple\\[1ex]
\lstinline!GMVIO(mesh).write_equation_systems ("out.gmv", es);!

\subsection{Add an additional vector, and project it on refined meshes}
\label{sec:add_vector}

Add a new vector to a system with \code{system.add\_vector("myvec");}. \\ Upon
mesh refinement, the vector can be projected onto the new mesh with 
\code{system.project\_vector(system.get\_vector("myvec"));}.


\section{Programming tips}
\label{sec:programming-tips}

\subsection{Autopointer}
\label{sec:autopointer}

Automatically take care of a pointer (safely delete it when it goes out of
scope) \code{AutoPtr<FEBase>} (\exmp{5})

\subsection{Scopes}
\label{sec:scopes}

Even in a very simple main program the there need to be scopes so that
the variables go out of scope before ending PetSc.

\begin{lstlisting}
libMesh::init (argc, argv);
  { 
    Mesh mesh(3);
  } 
return libMesh::close();
\end{lstlisting}

\end{document}



%%% Local Variables: 
%%% mode: latex
%%% TeX-master: t
%%% End: 
